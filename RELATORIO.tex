\documentclass[12pt,letterpaper]{article}
\usepackage[brazil]{babel}
\usepackage[utf8]{inputenc}
\usepackage{graphicx}
\usepackage{times}
\usepackage{url}
\usepackage{algorithm}
\usepackage{algorithmic}
\usepackage[bottom=2cm,top=2cm,left=2cm,right=2cm]{geometry}

\title{Relatório do EP3\\MAC0352 -- Redes de Computadores e Sistemas Distribuídos -- 2/2019}
\author{Julio Kenji Ueda ($9298281$), Ricardo Akira Tanaka ($9778856$)}
\date{}

\begin{document}
\maketitle

\section{Passo 0}
Na definição do protocolo OpenFlow, o que um switch faz toda
vez que ele recebe um pacote que ele nunca recebeu antes?\\

Um pacote nunca recebido não possui um fluxo definido, então o mesmo é enviado à controladora do switch. Esta pode adicionar um fluxo ao switch para tratar futuros pacotes semelhantes ou ignorar o pacote (drop).


\section{Passo 2}


Com o acesso à Internet funcionando em sua rede local, instale
na VM o programa \texttt{traceroute} usando \texttt{sudo apt install
traceroute} e escreva abaixo a saída do comando \texttt{sudo
traceroute -I www.inria.fr}. Pela saída do comando, a partir de qual
salto os pacotes alcançaram um roteador na Europa? Como você chegou a
essa conclusão?

\begin{tiny}
\begin{verbatim}
<traceroute to www.inria.fr (128.93.162.84), 30 hops max, 60 byte packets
 1  10.0.2.2 (10.0.2.2)  1.451 ms  0.892 ms  0.159 ms
 2  * * *
 3  * * *
 4  core-cce.uspnet.usp.br (143.107.255.225)  2.892 ms  3.381 ms  3.472 ms
 5  border1.uspnet.usp.br (143.107.251.29)  3.094 ms  2.877 ms  2.415 ms
 6  usp-sp.bkb.rnp.br (200.143.255.113)  3.152 ms  3.843 ms  3.572 ms
 7  br-rnp.redclara.net (200.0.204.213)  110.882 ms  111.045 ms  110.601 ms
 8  us-br.redclara.net (200.0.204.9)  111.203 ms  111.132 ms  111.039 ms
 9  redclara-gw.par.fr.geant.net (62.40.125.168)  212.854 ms  212.793 ms  212.873 ms
10  renater-lb1-gw.mx1.par.fr.geant.net (62.40.124.70)  212.777 ms  233.930 ms  233.771 ms
11  te1-1-inria-rtr-021.noc.renater.fr (193.51.177.107)  212.957 ms  212.683 ms  212.783 ms
12  inria-rocquencourt-te1-4-inria-rtr-021.noc.renater.fr (193.51.184.177)  213.021 ms  214.159 ms  214.016 ms
13  unit240-reth1-vfw-ext-dc1.inria.fr (192.93.122.19)  213.152 ms  212.698 ms  213.692 ms
14  ezp3.inria.fr (128.93.162.84)  213.233 ms  213.307 ms  212.836 ms>
\end{verbatim}
\end{tiny}

Na rede Eduroam da USP, a partir do salto 9 foi alcançado um servidor europeu,\newline \texttt{redclara-gw.par.fr.geant.net}
e indicado pela presença do domínio \texttt{fr} e pelo endereço IP.

\section{Passo 3 - Parte 1}


Execute o comando \texttt{iperf}, conforme descrito no
tutorial, antes de usar a opção \texttt{--switch user}, 5 vezes.
Escreva abaixo o valor médio e o intervalo de confiança da taxa
retornada (considere sempre o primeiro valor do vetor retornado).

\begin{verbatim}
*** Iperf: testing TCP bandwidth between h1 and h3
\end{verbatim}

\begin{center}
\begin{tabular}{ |c|c| } 
 \hline
 Teste & Gbits/sec \\
 \hline
 1 & 17.0 \\ 
 2 & 16.6 \\ 
 3 & 16.1 \\ 
 4 & 15.8 \\ 
 5 & 13.9 \\ 
 \hline
\end{tabular}
\end{center}

\begin{center}
\begin{tabular}{|c|c|}
 \hline
 Média (GBits/sec) & Intervalo de Confiança (GBits/sec)\\
 \hline
 15.88 & (14.39, 17.37) \\
 \hline
 \end{tabular}
\end{center}

\section{Passo 3 - Parte 2}


Execute o comando \texttt{iperf}, conforme descrito no
tutorial, com a opção \texttt{--switch user}, 5 vezes. Escreva abaixo
o valor médio e o intervalo de confiança da taxa retornada (considere
sempre o primeiro valor do vetor retornado). O resultado agora
corresponde a quantas vezes menos o da Seção anterior? Qual o motivo
dessa diferença?

\begin{verbatim}
*** Iperf: testing TCP bandwidth between h1 and h3
\end{verbatim}

\begin{center}
\begin{tabular}{ |c|c| } 
 \hline
 Teste & Mbits/sec \\
 \hline
 1 & 489 \\ 
 2 & 474 \\ 
 3 & 438 \\ 
 4 & 474 \\ 
 5 & 449 \\ 
 \hline
\end{tabular}
\end{center}

\begin{center}
\begin{tabular}{|c|c|}
 \hline
 Média (MBits/sec) & Intervalo de Confiança (MBits/sec)\\
 \hline
 464.8 & (439.03, 490.57) \\
 \hline
 \end{tabular}
\end{center}

Aproximadamente 34 vezes menos. Essa diferença de velocidade é causada pela passagem dos pacotes entre o kernel-space e o user-space. Enquanto na parte 1 o fluxo foi pré-definido e ocorria diretamente no switch, portanto somente dentro do kernel-space, na parte 2 a existência do controlador faz com que os pacotes tenham que passar do kernel-space para o user-space, onde está a controladora, para retornar ao kernel-space.

\section{Passo 4 - Parte 1}


Execute o comando \texttt{iperf}, conforme descrito no
tutorial, usando o controlador \texttt{of\_tutorial.py} original sem
modificação, 5 vezes. Escreva abaixo o valor médio e o intervalo de
confiança da taxa retornada (considere sempre o primeiro valor do
vetor retornado). O resultado agora
corresponde a quantas vezes menos o da Seção 3? Qual o motivo para
essa diferença? Use a saída do comando \texttt{tcpdump}, deixando
claro em quais computadores virtuais ele foi executado, para
justificar a sua resposta.

\begin{verbatim}
*** Iperf: testing TCP bandwidth between h1 and h3
\end{verbatim}

\begin{center}
\begin{tabular}{ |c|c|c| } 
 \hline
 Teste & Mbits/sec \\
 \hline
 1 & 21.3 \\ 
 2 & 31.3 \\ 
 3 & 31.5 \\ 
 4 & 22.4 \\ 
 5 & 31.5 \\ 
 \hline
\end{tabular}
\end{center}


\begin{center}
\begin{tabular}{|c|c|}
 \hline
 Média (MBits/sec) & Intervalo de Confiança (MBits/sec)\\
 \hline
 27.6 & (21.06, 34.14) \\
 \hline
 \end{tabular}
\end{center}

Aproximadamente 575 vezes menos. O comando \texttt{tcpdump} rodando no host 2 (\texttt{h2-eth0}) mostra pacotes originados e destinados a outros hosts sendo recebidos na interface de rede. Portanto fica claro que todo tráfego está passando pelo controlador, que atua como um hub, enviando os pacotes a todos os hosts conectados nele. A passagem pelo controlador causa a perda de desempenho.

\begin{tiny}
\begin{verbatim}
root@mininet-vm:~# tcpdump -XX -n -i h2-eth0

17:29:21.935960 IP 10.0.0.3.5001 > 10.0.0.1.54578: Flags [.], ack 10020185, win 1952, options [nop,nop,TS val 2801196 ecr 2801111], length 0
        0x0000:  0000 0000 0001 0000 0000 0003 0800 4500  ..............E.
        0x0010:  0034 947d 4000 4006 9243 0a00 0003 0a00  .4.}@.@..C......
        0x0020:  0001 1389 d532 1f84 3092 7d27 cf8b 8010  .....2..0.}'....
        0x0030:  07a0 593c 0000 0101 080a 002a be2c 002a  ..Y<.......*.,.*
        0x0040:  bdd7                                     ..
17:29:21.952843 IP 10.0.0.1.54578 > 10.0.0.3.5001: Flags [.], ack 2, win 58, options [nop,nop,TS val 2801220 ecr 2801210], length 0
        0x0000:  0000 0000 0003 0000 0000 0001 0800 4500  ..............E.
        0x0010:  0034 78c1 4000 4006 adff 0a00 0001 0a00  .4x.@.@.........
        0x0020:  0003 d532 1389 7d28 ea4c 1f84 3093 8010  ...2..}(.L..0...
        0x0030:  003a 4564 0000 0101 080a 002a be44 002a  .:Ed.......*.D.*
        0x0040:  be3a                                     .:
\end{verbatim}
\end{tiny}

\section{Passo 4 - Parte 2}


Execute o comando \texttt{iperf}, conforme descrito no
tutorial, usando o seu controlador \texttt{switch.py}, 5 vezes.
Escreva abaixo o valor médio e o intervalo de confiança da taxa
retornada (considere sempre o primeiro valor do vetor retornado). O
resultado agora corresponde a quantas vezes mais o da Seção anterior?
Qual o motivo dessa diferença? Use a saída do comando
\texttt{tcpdump}, deixando claro em quais computadores virtuais ele
foi executado, para justificar a sua resposta.

\begin{verbatim}
*** Iperf: testing TCP bandwidth between h1 and h3
\end{verbatim}

\begin{center}
\begin{tabular}{ |c|c| } 
 \hline
 Teste & Mbits/sec \\
 \hline
 1 & 30.4 \\ 
 2 & 31.0 \\ 
 3 & 22.7 \\ 
 4 & 30.2 \\ 
 5 & 28.1 \\ 
 \hline
\end{tabular}
\end{center}

\begin{center}
\begin{tabular}{|c|c|}
 \hline
 Média (MBits/sec) & Intervalo de Confiança (MBits/sec)\\
 \hline
 28.48 & (24.24, 32.72) \\
 \hline
 \end{tabular}
\end{center}

É ligeiramente maior, mas praticamente não há diferença pois todo o tráfego ainda passa pelo controlador, o que causa mau desempenho. O pequeno ganho do  \texttt{switch.py} é devido à memorização das portas dos endereços conhecidos, o que implica no envio dos pacotes somente ao destino e não a todos os hosts. Assim, as saídas do \texttt{tcpdump} estão presentes somente para os \texttt{hosts h1} e \texttt{h3} (é possível verificar as saídas do \texttt{tcpdump} direcionando a saída padrão para um arquivo, pois são muito extensas):

\begin{verbatim}
tcpdump no host h1 (20 primeiros pacotes)
\end{verbatim}

\begin{tiny}
\begin{verbatim}
root@mininet-vm:~# tcpdump-n -i h1-eth0 > dump_h1.txt
tcpdump: verbose output suppressed, use -v or -vv for full protocol decode
listening on h1-eth0, link-type EN10MB (Ethernet), capture size 262144 bytes

09:46:00.983889 ARP, Request who-has 10.0.0.3 tell 10.0.0.1, length 28
09:46:00.991436 ARP, Reply 10.0.0.3 is-at 00:00:00:00:00:03, length 28
09:46:00.991442 IP 10.0.0.1.45992 > 10.0.0.3.5001: Flags [S], seq 1032372313, win 29200,
    options [mss 1460,sackOK,TS val 875252 ecr 0,nop,wscale 9], length 0
09:46:00.992496 IP 10.0.0.3.5001 > 10.0.0.1.45992: Flags [S.], seq 798501729, ack 1032372314, win 28960,
    options [mss 1460,sackOK,TS val 875254 ecr 875252,nop,wscale 9], length 0
09:46:00.992511 IP 10.0.0.1.45992 > 10.0.0.3.5001: Flags [.], ack 1, win 58, 
    options [nop,nop,TS val 875254 ecr 875254], length 0
09:46:00.993366 IP 10.0.0.1.45992 > 10.0.0.3.5001: Flags [F.], seq 1, ack 1, win 58, options [nop,nop,TS val 875254 ecr 875254], length 0
09:46:00.995805 IP 10.0.0.1.45994 > 10.0.0.3.5001: Flags [S], seq 2225526446, win 29200,
    options [mss 1460,sackOK,TS val 875254 ecr 0,nop,wscale 9], length 0
09:46:01.003606 IP 10.0.0.3.5001 > 10.0.0.1.45992: Flags [.], ack 2, win 57, options [nop,nop,TS val 875257 ecr 875254], length 0
09:46:01.010078 IP 10.0.0.3.5001 > 10.0.0.1.45994: Flags [S.], seq 4120615762, ack 2225526447, win 28960,
    options [mss 1460,sackOK,TS val 875257 ecr 875254,nop,wscale 9], length 0
09:46:01.010092 IP 10.0.0.1.45994 > 10.0.0.3.5001: Flags [.], ack 1, win 58, options [nop,nop,TS val 875259 ecr 875257], length 0
09:46:01.010321 IP 10.0.0.1.45994 > 10.0.0.3.5001: Flags [P.], seq 1:25, ack 1, win 58, options [nop,nop,TS val 875259 ecr 875257], length 24
09:46:01.010543 IP 10.0.0.1.45994 > 10.0.0.3.5001: Flags [.], seq 25:2921, ack 1, win 58, options [nop,nop,TS val 875259 ecr 875257], length 2896
09:46:01.010562 IP 10.0.0.1.45994 > 10.0.0.3.5001: Flags [.], seq 2921:5817, ack 1, win 58, options [nop,nop,TS val 875259 ecr 875257], length 2896
09:46:01.010569 IP 10.0.0.1.45994 > 10.0.0.3.5001: Flags [.], seq 5817:8713, ack 1, win 58, options [nop,nop,TS val 875259 ecr 875257], length 2896
09:46:01.010576 IP 10.0.0.1.45994 > 10.0.0.3.5001: Flags [.], seq 8713:11609, ack 1, win 58, options [nop,nop,TS val 875259 ecr 875257], length 2896
09:46:01.010582 IP 10.0.0.1.45994 > 10.0.0.3.5001: Flags [.], seq 11609:13057, ack 1, win 58, options [nop,nop,TS val 875259 ecr 875257], length 1448
09:46:01.012396 IP 10.0.0.3.5001 > 10.0.0.1.45992: Flags [F.], seq 1, ack 2, win 57, options [nop,nop,TS val 875259 ecr 875254], length 0
09:46:01.012409 IP 10.0.0.1.45992 > 10.0.0.3.5001: Flags [.], ack 2, win 58, options [nop,nop,TS val 875260 ecr 875259], length 0
09:46:01.038680 IP 10.0.0.1.45994 > 10.0.0.3.5001: Flags [.], seq 13057:14505, ack 1, win 58, options [nop,nop,TS val 875267 ecr 875257], length 1448
09:46:01.051993 IP 10.0.0.3.5001 > 10.0.0.1.45994: Flags [.], ack 25, win 57, options [nop,nop,TS val 875267 ecr 875259], length 0
\end{verbatim}
\end{tiny}

\begin{verbatim}
tcpdump no host h2 (vazio)
\end{verbatim}

\begin{tiny}
\begin{verbatim}
root@mininet-vm:~# tcpdump -n -i h2-eth0 > dump_h2.txt
tcpdump: verbose output suppressed, use -v or -vv for full protocol decode
listening on h2-eth0, link-type EN10MB (Ethernet), capture size 262144 bytes
\end{verbatim}
\end{tiny}

\begin{verbatim}
tcpdump no host h3 (20 primeiros pacotes)
\end{verbatim}

\begin{tiny}
\begin{verbatim}
root@mininet-vm:~# tcpdump -n -i h3-eth0 > dump_h3.txt
tcpdump: verbose output suppressed, use -v or -vv for full protocol decode
listening on h3-eth0, link-type EN10MB (Ethernet), capture size 262144 bytes

09:46:00.990693 ARP, Request who-has 10.0.0.3 tell 10.0.0.1, length 28
09:46:00.990713 ARP, Reply 10.0.0.3 is-at 00:00:00:00:00:03, length 28
09:46:00.991981 IP 10.0.0.1.45992 > 10.0.0.3.5001: Flags [S], seq 1032372313, win 29200,
    options [mss 1460,sackOK,TS val 875252 ecr 0,nop,wscale 9], length 0
09:46:00.991994 IP 10.0.0.3.5001 > 10.0.0.1.45992: Flags [S.], seq 798501729, ack 1032372314, win 28960,
    options [mss 1460,sackOK,TS val 875254 ecr 875252,nop,wscale 9], length 0
09:46:00.992956 IP 10.0.0.1.45992 > 10.0.0.3.5001: Flags [.], ack 1, win 58, options [nop,nop,TS val 875254 ecr 875254], length 0
09:46:00.996804 IP 10.0.0.1.45992 > 10.0.0.3.5001: Flags [F.], seq 1, ack 1, win 58, options [nop,nop,TS val 875254 ecr 875254], length 0
09:46:00.999825 IP 10.0.0.3.5001 > 10.0.0.1.45992: Flags [.], ack 2, win 57, options [nop,nop,TS val 875257 ecr 875254], length 0
09:46:01.000238 IP 10.0.0.1.45994 > 10.0.0.3.5001: Flags [S], seq 2225526446, win 29200,
    options [mss 1460,sackOK,TS val 875254 ecr 0,nop,wscale 9], length 0
09:46:01.000265 IP 10.0.0.3.5001 > 10.0.0.1.45994: Flags [S.], seq 4120615762, ack 2225526447, win 28960,
    options [mss 1460,sackOK,TS val 875257 ecr 875254,nop,wscale 9], length 0
09:46:01.009917 IP 10.0.0.3.5001 > 10.0.0.1.45992: Flags [F.], seq 1, ack 2, win 57, options [nop,nop,TS val 875259 ecr 875254], length 0
09:46:01.013167 IP 10.0.0.1.45994 > 10.0.0.3.5001: Flags [.], ack 1, win 58, options [nop,nop,TS val 875259 ecr 875257], length 0
09:46:01.039002 IP 10.0.0.1.45994 > 10.0.0.3.5001: Flags [P.], seq 1:25, ack 1, win 58, options [nop,nop,TS val 875259 ecr 875257], length 24
09:46:01.039039 IP 10.0.0.1.45994 > 10.0.0.3.5001: Flags [.], seq 25:1473, ack 1, win 58, options [nop,nop,TS val 875259 ecr 875257], length 1448
09:46:01.039058 IP 10.0.0.1.45994 > 10.0.0.3.5001: Flags [.], seq 1473:2921, ack 1, win 58, options [nop,nop,TS val 875259 ecr 875257], length 1448
09:46:01.039074 IP 10.0.0.1.45994 > 10.0.0.3.5001: Flags [.], seq 2921:4369, ack 1, win 58, options [nop,nop,TS val 875259 ecr 875257], length 1448
09:46:01.039091 IP 10.0.0.1.45994 > 10.0.0.3.5001: Flags [.], seq 4369:5817, ack 1, win 58, options [nop,nop,TS val 875259 ecr 875257], length 1448
09:46:01.039114 IP 10.0.0.1.45994 > 10.0.0.3.5001: Flags [.], seq 5817:7265, ack 1, win 58, options [nop,nop,TS val 875259 ecr 875257], length 1448
09:46:01.039131 IP 10.0.0.1.45994 > 10.0.0.3.5001: Flags [.], seq 7265:8713, ack 1, win 58, options [nop,nop,TS val 875259 ecr 875257], length 1448
09:46:01.039146 IP 10.0.0.1.45994 > 10.0.0.3.5001: Flags [.], seq 8713:10161, ack 1, win 58, options [nop,nop,TS val 875259 ecr 875257], length 1448
09:46:01.039160 IP 10.0.0.1.45994 > 10.0.0.3.5001: Flags [.], seq 10161:11609, ack 1, win 58, options [nop,nop,TS val 875259 ecr 875257], length 1448
\end{verbatim}
\end{tiny}

\section{Passo 4 - Parte 3}

Execute o comando \texttt{iperf}, conforme descrito no
tutorial, usando o seu controlador \texttt{switch.py} melhorado, 5
vezes. Escreva abaixo o valor médio e o intervalo de confiança da taxa
retornada (considere sempre o primeiro valor do vetor retornado). O
resultado agora corresponde a quantas vezes mais o da Seção anterior?
Qual o motivo dessa diferença? Use a saída do comando
\texttt{tcpdump}, deixando claro em quais computadores virtuais ele
foi executado, e saídas do comando \texttt{sudo ovs-ofctl}, com os
devidos parâmetros, para justificar a sua resposta.

\begin{center}
\begin{tabular}{ |c|c| } 
 \hline
 Teste & Gbits/sec \\
 \hline
 1 & 23.0 \\ 
 2 & 24.4 \\ 
 3 & 23.1 \\ 
 4 & 24.5 \\ 
 5 & 24.0 \\ 
 \hline
\end{tabular}
\end{center}

\begin{center}
\begin{tabular}{|c|c|}
 \hline
 Média (GBits/sec) & Intervalo de Confiança (GBits/sec)\\
 \hline
 23.8 & (22.92, 24.68) \\
 \hline
 \end{tabular}
\end{center}

Aproximadamente 835 vezes maior, e esta diferença é devido ao envio dos fluxos conhecidos pelo controlador ao \texttt{switch}. Agora, os pacotes recebidos por uma porta com destino conhecido são enviados à porta correspondente sem passar pelo controlador, implicando em ganho de desempenho.

O \texttt{fluxo do switch s1} mostra todos os fluxos conhecidos e enviados pelo controlador ao \texttt{switch}. É possível perceber que há apenas fluxos entres os \texttt{hosts 10.0.0.1} e \texttt{10.0.0.3} (\texttt{h1} e \texttt{h3}). Também é possível perceber que o \texttt{host h2} não viu nenhum pacote pela saída do \texttt{tcpdump}:

\begin{verbatim}
fluxo do switch s1
\end{verbatim}

\begin{tiny}
\begin{verbatim}
mininet@mininet-vm:~$ sudo ovs-ofctl dump-flows s1

NXST_FLOW reply (xid=0x4):
 cookie=0x0, duration=409.631s, table=0, n_packets=43761, n_bytes=2490634810, idle_age=404, tcp,vlan_tci=0x0000,
 dl_src=00:00:00:00:00:01,dl_dst=00:00:00:00:00:03,nw_src=10.0.0.1,nw_dst=10.0.0.3,nw_tos=0,tp_src=46016,tp_dst=5001 actions=output:3
 cookie=0x0, duration=409.715s, table=0, n_packets=3, n_bytes=198, idle_age=409, tcp,vlan_tci=0x0000,
 dl_src=00:00:00:00:00:01,dl_dst=00:00:00:00:00:03,nw_src=10.0.0.1,nw_dst=10.0.0.3,nw_tos=16,tp_src=46014,tp_dst=5001 actions=output:3
 cookie=0x0, duration=409.559s, table=0, n_packets=10112, n_bytes=667392, idle_age=404, tcp,vlan_tci=0x0000,
 dl_src=00:00:00:00:00:03,dl_dst=00:00:00:00:00:01,nw_src=10.0.0.3,nw_dst=10.0.0.1,nw_tos=0,tp_src=5001,tp_dst=46016 actions=output:1
 cookie=0x0, duration=409.669s, table=0, n_packets=1, n_bytes=66, idle_age=409, tcp,vlan_tci=0x0000,
 dl_src=00:00:00:00:00:03,dl_dst=00:00:00:00:00:01,nw_src=10.0.0.3,nw_dst=10.0.0.1,nw_tos=0,tp_src=5001,tp_dst=46014 actions=output:1
 cookie=0x0, duration=409.752s, table=0, n_packets=0, n_bytes=0, idle_age=409, arp,vlan_tci=0x0000,
 dl_src=00:00:00:00:00:03,dl_dst=00:00:00:00:00:01,arp_spa=10.0.0.3,arp_tpa=10.0.0.1,arp_op=2 actions=output:1

\end{verbatim}
\end{tiny}

\begin{verbatim}
tcpdump do host h1 (20 primeiros pacotes)
\end{verbatim}

\begin{tiny}
\begin{verbatim}
root@mininet-vm:~# tcpdump -n -i h1-eth0 > dump_h1.txt
tcpdump: verbose output suppressed, use -v or -vv for full protocol decode
listening on h1-eth0, link-type EN10MB (Ethernet), capture size 262144 bytes

09:56:20.435536 ARP, Request who-has 10.0.0.3 tell 10.0.0.1, length 28
09:56:20.437085 ARP, Reply 10.0.0.3 is-at 00:00:00:00:00:03, length 28
09:56:20.437090 IP 10.0.0.1.46014 > 10.0.0.3.5001: Flags [S], seq 1455017156, win 29200,
    options [mss 1460,sackOK,TS val 1030115 ecr 0,nop,wscale 9], length 0
09:56:20.505075 IP 10.0.0.3.5001 > 10.0.0.1.46014: Flags [S.], seq 2948560447, ack 1455017157, win 28960,
    options [mss 1460,sackOK,TS val 1030126 ecr 1030115,nop,wscale 9], length 0
09:56:20.505112 IP 10.0.0.1.46014 > 10.0.0.3.5001: Flags [.], ack 1, win 58, options [nop,nop,TS val 1030133 ecr 1030126], length 0
09:56:20.506125 IP 10.0.0.1.46014 > 10.0.0.3.5001: Flags [F.], seq 1, ack 1, win 58, options [nop,nop,TS val 1030133 ecr 1030126], length 0
09:56:20.520749 IP 10.0.0.1.46016 > 10.0.0.3.5001: Flags [S], seq 1231945580, win 29200,
    options [mss 1460,sackOK,TS val 1030137 ecr 0,nop,wscale 9], length 0
09:56:20.520922 IP 10.0.0.3.5001 > 10.0.0.1.46014: Flags [.], ack 2, win 57, options [nop,nop,TS val 1030134 ecr 1030133], length 0
09:56:20.521224 IP 10.0.0.3.5001 > 10.0.0.1.46014: Flags [F.], seq 1, ack 2, win 57, options [nop,nop,TS val 1030137 ecr 1030133], length 0
09:56:20.521236 IP 10.0.0.1.46014 > 10.0.0.3.5001: Flags [.], ack 2, win 58, options [nop,nop,TS val 1030137 ecr 1030137], length 0
09:56:20.588779 IP 10.0.0.3.5001 > 10.0.0.1.46016: Flags [S.], seq 2795696301, ack 1231945581, win 28960,
    options [mss 1460,sackOK,TS val 1030147 ecr 1030137,nop,wscale 9], length 0
09:56:20.588815 IP 10.0.0.1.46016 > 10.0.0.3.5001: Flags [.], ack 1, win 58, options [nop,nop,TS val 1030154 ecr 1030147], length 0
09:56:20.590129 IP 10.0.0.1.46016 > 10.0.0.3.5001: Flags [P.], seq 1:25, ack 1, win 58, options [nop,nop,TS val 1030154 ecr 1030147], length 24
09:56:20.591467 IP 10.0.0.1.46016 > 10.0.0.3.5001: Flags [.], seq 25:2921, ack 1, win 58, options [nop,nop,TS val 1030155 ecr 1030147], length 2896
09:56:20.591524 IP 10.0.0.1.46016 > 10.0.0.3.5001: Flags [.], seq 2921:5817, ack 1, win 58, options [nop,nop,TS val 1030155 ecr 1030147], length 2896
09:56:20.591550 IP 10.0.0.1.46016 > 10.0.0.3.5001: Flags [.], seq 5817:8713, ack 1, win 58, options [nop,nop,TS val 1030155 ecr 1030147], length 2896
09:56:20.591571 IP 10.0.0.1.46016 > 10.0.0.3.5001: Flags [.], seq 8713:11609, ack 1, win 58, options [nop,nop,TS val 1030155 ecr 1030147], length 2896
09:56:20.591589 IP 10.0.0.1.46016 > 10.0.0.3.5001: Flags [.], seq 11609:13057, ack 1, win 58, options [nop,nop,TS val 1030155 ecr 1030147], length 1448
09:56:20.630758 IP 10.0.0.3.5001 > 10.0.0.1.46016: Flags [.], ack 25, win 57, options [nop,nop,TS val 1030154 ecr 1030154], length 0
09:56:20.630776 IP 10.0.0.1.46016 > 10.0.0.3.5001: Flags [.], seq 13057:15953, ack 1, win 58, options [nop,nop,TS val 1030165 ecr 1030154], length 2896
\end{verbatim}
\end{tiny}

\begin{verbatim}
tcpdump no host h2 (vazio)
\end{verbatim}

\begin{tiny}
\begin{verbatim}
root@mininet-vm:~# tcpdump -n -i h2-eth0 > dump_h2.txt
tcpdump: verbose output suppressed, use -v or -vv for full protocol decode
listening on h2-eth0, link-type EN10MB (Ethernet), capture size 262144 bytes
\end{verbatim}
\end{tiny}

\begin{verbatim}
tcpdump do host h3 (20 primeiros pacotes)
\end{verbatim}

\begin{tiny}
\begin{verbatim}
root@mininet-vm:~# tcpdump -n -i h3-eth0 > dump_h3.txt
tcpdump: verbose output suppressed, use -v or -vv for full protocol decode
listening on h1-eth0, link-type EN10MB (Ethernet), capture size 262144 bytes

09:56:20.436575 ARP, Request who-has 10.0.0.3 tell 10.0.0.1, length 28
09:56:20.436588 ARP, Reply 10.0.0.3 is-at 00:00:00:00:00:03, length 28
09:56:20.474823 IP 10.0.0.1.46014 > 10.0.0.3.5001: Flags [S], seq 1455017156, win 29200,
    options [mss 1460,sackOK,TS val 1030115 ecr 0,nop,wscale 9], length 0
09:56:20.474853 IP 10.0.0.3.5001 > 10.0.0.1.46014: Flags [S.], seq 2948560447, ack 1455017157, win 28960,
    options [mss 1460,sackOK,TS val 1030126 ecr 1030115,nop,wscale 9], length 0
09:56:20.505381 IP 10.0.0.1.46014 > 10.0.0.3.5001: Flags [.], ack 1, win 58, options [nop,nop,TS val 1030133 ecr 1030126], length 0
09:56:20.506136 IP 10.0.0.1.46014 > 10.0.0.3.5001: Flags [F.], seq 1, ack 1, win 58, options [nop,nop,TS val 1030133 ecr 1030126], length 0
09:56:20.508192 IP 10.0.0.3.5001 > 10.0.0.1.46014: Flags [.], ack 2, win 57, options [nop,nop,TS val 1030134 ecr 1030133], length 0
09:56:20.521109 IP 10.0.0.3.5001 > 10.0.0.1.46014: Flags [F.], seq 1, ack 2, win 57, options [nop,nop,TS val 1030137 ecr 1030133], length 0
09:56:20.521240 IP 10.0.0.1.46014 > 10.0.0.3.5001: Flags [.], ack 2, win 58, options [nop,nop,TS val 1030137 ecr 1030137], length 0
09:56:20.558838 IP 10.0.0.1.46016 > 10.0.0.3.5001: Flags [S], seq 1231945580, win 29200,
    options [mss 1460,sackOK,TS val 1030137 ecr 0,nop,wscale 9], length 0
09:56:20.558865 IP 10.0.0.3.5001 > 10.0.0.1.46016: Flags [S.], seq 2795696301, ack 1231945581, win 28960,
    options [mss 1460,sackOK,TS val 1030147 ecr 1030137,nop,wscale 9], length 0
09:56:20.589093 IP 10.0.0.1.46016 > 10.0.0.3.5001: Flags [.], ack 1, win 58, options [nop,nop,TS val 1030154 ecr 1030147], length 0
09:56:20.590142 IP 10.0.0.1.46016 > 10.0.0.3.5001: Flags [P.], seq 1:25, ack 1, win 58, options [nop,nop,TS val 1030154 ecr 1030147], length 24
09:56:20.590154 IP 10.0.0.3.5001 > 10.0.0.1.46016: Flags [.], ack 25, win 57, options [nop,nop,TS val 1030154 ecr 1030154], length 0
09:56:20.591482 IP 10.0.0.1.46016 > 10.0.0.3.5001: Flags [.], seq 25:2921, ack 1, win 58, options [nop,nop,TS val 1030155 ecr 1030147], length 2896
09:56:20.591494 IP 10.0.0.3.5001 > 10.0.0.1.46016: Flags [.], ack 2921, win 68, options [nop,nop,TS val 1030155 ecr 1030155], length 0
09:56:20.591529 IP 10.0.0.1.46016 > 10.0.0.3.5001: Flags [.], seq 2921:5817, ack 1, win 58, options [nop,nop,TS val 1030155 ecr 1030147], length 2896
09:56:20.591537 IP 10.0.0.3.5001 > 10.0.0.1.46016: Flags [.], ack 5817, win 80, options [nop,nop,TS val 1030155 ecr 1030155], length 0
09:56:20.591553 IP 10.0.0.1.46016 > 10.0.0.3.5001: Flags [.], seq 5817:8713, ack 1, win 58, options [nop,nop,TS val 1030155 ecr 1030147], length 2896
09:56:20.591560 IP 10.0.0.3.5001 > 10.0.0.1.46016: Flags [.], ack 8713, win 91, options [nop,nop,TS val 1030155 ecr 1030155], length 0
\end{verbatim}
\end{tiny}

\section{Passo 5}

Explique a lógica implementada no seu controlador \texttt{firewall.py} e mostre saídas de comandos que comprovem que ele está de fato funcionando (saídas dos comandos \texttt{tcpdump}, \texttt{sudo ovs-ofctl}, \texttt{nc}, \texttt{iperf} e \texttt{telnet} são recomendadas) \\

O Firewall lê as regras a partir de um arquivo externo (a definição das regras no arquivo externo estão no arquivo \texttt{LEIAME.pdf}), e todas as regras são enviadas ao \texttt{switch}. Uma regra é composta por 4 parâmetros (endereço de origem, destino, porta e protocolo) e o Firewall é composto por várias regras. Um pacote é bloqueado se casa com os parâmetros definidos por qualquer regra. \\

\textbf{Regra de bloqueio por endereço: bloquear pacotes entre o endereço de origem 10.0.0.1 e destino 10.0.03}

Devemos mostrar que não há fluxo de pacotes direcionado entre \texttt{h1} e \texttt{h3}:

\begin{verbatim}
iperf do h1 como cliente tentando acessar h3:
\end{verbatim}

\begin{tiny}
\begin{verbatim}
root@mininet-vm:~# iperf -c 10.0.0.3
connect failed: Connection timed out
root@mininet-vm:~# 
\end{verbatim}
\end{tiny}

\begin{verbatim}
iperf do h2 como cliente tentando acessar h3:
\end{verbatim}

\begin{tiny}
\begin{verbatim}
root@mininet-vm:~# iperf -c 10.0.0.3
------------------------------------------------------------
Client connecting to 10.0.0.3, TCP port 5001
TCP window size: 85.3 KByte (default)
------------------------------------------------------------
[ 15] local 10.0.0.2 port 44088 connected with 10.0.0.3 port 5001
[ ID] Interval       Transfer     Bandwidth
[ 15]  0.0-10.0 sec  27.7 GBytes  23.8 Gbits/sec
root@mininet-vm:~# 
\end{verbatim}
\end{tiny}

\begin{verbatim}
iperf do h3 como servidor:
\end{verbatim}

\begin{tiny}
\begin{verbatim}
root@mininet-vm:~# iperf -s
------------------------------------------------------------
Server listening on TCP port 5001
TCP window size: 85.3 KByte (default)
------------------------------------------------------------
[ 16] local 10.0.0.3 port 5001 connected with 10.0.0.2 port 44088
[ ID] Interval       Transfer     Bandwidth
[ 16]  0.0-10.0 sec  27.7 GBytes  23.7 Gbits/sec
\end{verbatim}
\end{tiny}

Criando um ambiente cliente/servidor com o \texttt{iperf}, o cliente \texttt{h1} não consegue realizar a conexão com o servidor \texttt{h3}, pois todos os pacotes são bloqueados. O cliente \texttt{h2} consegue realizar a conexão sem problemas.

\begin{verbatim}
pingall:
\end{verbatim}

\begin{tiny}
\begin{verbatim}
mininet> pingall
*** Ping: testing ping reachability
h1 -> h2 X 
h2 -> h1 h3 
h3 -> X h2 
*** Results: 33% dropped (4/6 received)
mininet> 
\end{verbatim}
\end{tiny}

Realizando o \texttt{pingall} entre todos os \texttt{hosts}, é possível verificar pela saída \texttt{tcpdump no host h1} que:
\begin{itemize}
    \item \texttt{h1} envia um \texttt{echo request} para \texttt{h2} e recebe um \texttt{echo reply} do mesmo.
    \item \texttt{h1} envia um \texttt{echo request} para \texttt{h3}, mas não recebe o \texttt{echo reply}.
    \item \texttt{h1} recebe \texttt{echo request} de \texttt{h2} e \texttt{h3} e envia \texttt{echo reply} para ambos.
\end{itemize}

\begin{verbatim}
tcpdump no host h1
\end{verbatim}

\begin{tiny}
\begin{verbatim}
root@mininet-vm:~# tcpdump -n -i h1-eth0
tcpdump: verbose output suppressed, use -v or -vv for full protocol decode
listening on h1-eth0, link-type EN10MB (Ethernet), capture size 262144 bytes
12:28:36.869134 IP 10.0.0.1 > 10.0.0.2: ICMP echo request, id 8144, seq 1, length 64
12:28:36.869903 IP 10.0.0.2 > 10.0.0.1: ICMP echo reply, id 8144, seq 1, length 64
12:28:36.877940 IP 10.0.0.1 > 10.0.0.3: ICMP echo request, id 8145, seq 1, length 64
12:28:41.874790 ARP, Request who-has 10.0.0.2 tell 10.0.0.1, length 28
12:28:41.875105 ARP, Request who-has 10.0.0.1 tell 10.0.0.2, length 28
12:28:41.875114 ARP, Reply 10.0.0.1 is-at 00:00:00:00:00:01, length 28
12:28:41.875532 ARP, Reply 10.0.0.2 is-at 00:00:00:00:00:02, length 28
12:28:41.890470 ARP, Request who-has 10.0.0.3 tell 10.0.0.1, length 28
12:28:41.890882 ARP, Reply 10.0.0.3 is-at 00:00:00:00:00:03, length 28
12:28:46.884129 IP 10.0.0.2 > 10.0.0.1: ICMP echo request, id 8148, seq 1, length 64
12:28:46.884155 IP 10.0.0.1 > 10.0.0.2: ICMP echo reply, id 8148, seq 1, length 64
12:28:46.897372 IP 10.0.0.3 > 10.0.0.1: ICMP echo request, id 8150, seq 1, length 64
12:28:46.897383 IP 10.0.0.1 > 10.0.0.3: ICMP echo reply, id 8150, seq 1, length 64
12:28:51.906856 ARP, Request who-has 10.0.0.1 tell 10.0.0.3, length 28
12:28:51.906875 ARP, Reply 10.0.0.1 is-at 00:00:00:00:00:01, length 28
\end{verbatim}
\end{tiny} 

É possível perceber pela saída \texttt{tcpdump no host h2} que:
\begin{itemize}
    \item \texttt{h2} recebe \texttt{echo request} de \texttt{h1} e \texttt{h3} e envia \texttt{echo reply} para ambos.
    \item \texttt{h2} envia \texttt{echo request} para \texttt{h1} e \texttt{h3} e recebe \texttt{echo reply} de ambos.
\end{itemize}

\begin{verbatim}
tcpdump no host h2
\end{verbatim}

\begin{tiny}
\begin{verbatim}
root@mininet-vm:~# tcpdump -n -i h2-eth0
tcpdump: verbose output suppressed, use -v or -vv for full protocol decode
listening on h2-eth0, link-type EN10MB (Ethernet), capture size 262144 bytes
12:28:36.869367 IP 10.0.0.1 > 10.0.0.2: ICMP echo request, id 8144, seq 1, length 64
12:28:36.869403 IP 10.0.0.2 > 10.0.0.1: ICMP echo reply, id 8144, seq 1, length 64
12:28:41.874819 ARP, Request who-has 10.0.0.1 tell 10.0.0.2, length 28
12:28:41.875034 ARP, Request who-has 10.0.0.2 tell 10.0.0.1, length 28
12:28:41.875052 ARP, Reply 10.0.0.2 is-at 00:00:00:00:00:02, length 28
12:28:41.875516 ARP, Reply 10.0.0.1 is-at 00:00:00:00:00:01, length 28
12:28:46.883952 IP 10.0.0.2 > 10.0.0.1: ICMP echo request, id 8148, seq 1, length 64
12:28:46.884540 IP 10.0.0.1 > 10.0.0.2: ICMP echo reply, id 8148, seq 1, length 64
12:28:46.893729 IP 10.0.0.2 > 10.0.0.3: ICMP echo request, id 8149, seq 1, length 64
12:28:46.893975 IP 10.0.0.3 > 10.0.0.2: ICMP echo reply, id 8149, seq 1, length 64
12:28:51.906608 ARP, Request who-has 10.0.0.3 tell 10.0.0.2, length 28
12:28:51.906929 ARP, Request who-has 10.0.0.2 tell 10.0.0.3, length 28
12:28:51.906940 ARP, Reply 10.0.0.2 is-at 00:00:00:00:00:02, length 28
12:28:51.907436 ARP, Reply 10.0.0.3 is-at 00:00:00:00:00:03, length 28
12:28:56.902982 IP 10.0.0.3 > 10.0.0.2: ICMP echo request, id 8153, seq 1, length 64
12:28:56.903009 IP 10.0.0.2 > 10.0.0.3: ICMP echo reply, id 8153, seq 1, length 64
\end{verbatim}
\end{tiny}

É possível perceber pela saída \texttt{tcpdump no host h3} que:
\begin{itemize}
    \item \texttt{h3} recebe \texttt{echo request} de \texttt{h2} e envia o \texttt{echo reply} para o mesmo.
    \item \texttt{h3} envia \texttt{echo request} para \texttt{h2} e recebe \texttt{echo reply} do mesmo.
    \item \texttt{h3} não recebe o \texttt{echo request} de \texttt{h1}.
    \item \texttt{h3} envia o \texttt{echo request} para \texttt{h1}, mas não recebe o \texttt{echo reply} do mesmo. 
\end{itemize}

\begin{verbatim}
tcpdump no host h3
\end{verbatim}

\begin{tiny}
\begin{verbatim}
root@mininet-vm:~# tcpdump -n -i h3-eth0
tcpdump: verbose output suppressed, use -v or -vv for full protocol decode
listening on h3-eth0, link-type EN10MB (Ethernet), capture size 262144 bytes
12:28:41.890631 ARP, Request who-has 10.0.0.3 tell 10.0.0.1, length 28
12:28:41.890645 ARP, Reply 10.0.0.3 is-at 00:00:00:00:00:03, length 28
12:28:46.893824 IP 10.0.0.2 > 10.0.0.3: ICMP echo request, id 8149, seq 1, length 64
12:28:46.893836 IP 10.0.0.3 > 10.0.0.2: ICMP echo reply, id 8149, seq 1, length 64
12:28:46.897290 IP 10.0.0.3 > 10.0.0.1: ICMP echo request, id 8150, seq 1, length 64
12:28:51.906633 ARP, Request who-has 10.0.0.2 tell 10.0.0.3, length 28
12:28:51.906643 ARP, Request who-has 10.0.0.1 tell 10.0.0.3, length 28
12:28:51.906980 ARP, Request who-has 10.0.0.3 tell 10.0.0.2, length 28
12:28:51.906989 ARP, Reply 10.0.0.3 is-at 00:00:00:00:00:03, length 28
12:28:51.907419 ARP, Reply 10.0.0.2 is-at 00:00:00:00:00:02, length 28
12:28:51.907444 ARP, Reply 10.0.0.1 is-at 00:00:00:00:00:01, length 28
12:28:56.902809 IP 10.0.0.3 > 10.0.0.2: ICMP echo request, id 8153, seq 1, length 64
12:28:56.903404 IP 10.0.0.2 > 10.0.0.3: ICMP echo reply, id 8153, seq 1, length 64
\end{verbatim}
\end{tiny}

Portanto, os pacotes de origem \texttt{h1} e destino \texttt{h3} estão bloqueados.\\

\textbf{Regra de bloqueio por protocolo: bloquear todos os pacotes com protocolo UDP}

Devemos mostrar que qualquer pacote UDP independentemente do \texttt{host} ou da \texttt{porta} é bloqueado:

\begin{verbatim}
iperf do h1 como servidor daemon TCP e UDP:
\end{verbatim}

\begin{tiny}
\begin{verbatim}
root@mininet-vm:~# iperf -s -D
root@mininet-vm:~# Running Iperf Server as a daemon
The Iperf daemon process ID : 12235

root@mininet-vm:~# iperf -u -s -D
root@mininet-vm:~# Running Iperf Server as a daemon
The Iperf daemon process ID : 12242

root@mininet-vm:~# 
\end{verbatim}
\end{tiny}

É possível perceber pelas saídas do \texttt{iperf} de \texttt{h2} e \texttt{h3} que ambos os clientes conseguem conectar-se ao servidor através do protocolo TCP, mas não com UDP (a última linha exibe um \textit{WARNING} sobre o não recebimento do pacote) utilizando a mesma porta (5001).

\begin{verbatim}
iperf do h2 como cliente TCP e depois UDP tentando acessar h1:
\end{verbatim}

\begin{tiny}
\begin{verbatim}
root@mininet-vm:~# iperf -c 10.0.0.1
------------------------------------------------------------
Client connecting to 10.0.0.1, TCP port 5001
TCP window size: 85.3 KByte (default)
------------------------------------------------------------
[ 15] local 10.0.0.2 port 37838 connected with 10.0.0.1 port 5001
[ ID] Interval       Transfer     Bandwidth
[ 15]  0.0-10.0 sec  12.1 GBytes  10.4 Gbits/sec
root@mininet-vm:~# iperf -u -c 10.0.0.1
------------------------------------------------------------
Client connecting to 10.0.0.1, UDP port 5001
Sending 1470 byte datagrams
UDP buffer size:  208 KByte (default)
------------------------------------------------------------
[ 15] local 10.0.0.2 port 33262 connected with 10.0.0.1 port 5001
[ ID] Interval       Transfer     Bandwidth
[ 15]  0.0-10.0 sec  1.25 MBytes  1.05 Mbits/sec
[ 15] Sent 893 datagrams
[ 15] WARNING: did not receive ack of last datagram after 10 tries.
root@mininet-vm:~# 
\end{verbatim}
\end{tiny}

\begin{verbatim}
iperf do h3 como cliente TCP e depois UDP tentando acessar h1:
\end{verbatim}

\begin{tiny}
\begin{verbatim}
root@mininet-vm:~# iperf -c 10.0.0.1
------------------------------------------------------------
Client connecting to 10.0.0.1, TCP port 5001
TCP window size: 85.3 KByte (default)
------------------------------------------------------------
[ 15] local 10.0.0.3 port 36012 connected with 10.0.0.1 port 5001
[ ID] Interval       Transfer     Bandwidth
[ 15]  0.0-10.0 sec  8.99 GBytes  7.72 Gbits/sec
root@mininet-vm:~# iperf -u -c 10.0.0.1
------------------------------------------------------------
Client connecting to 10.0.0.1, UDP port 5001
Sending 1470 byte datagrams
UDP buffer size:  208 KByte (default)
------------------------------------------------------------
[ 15] local 10.0.0.3 port 34679 connected with 10.0.0.1 port 5001
[ ID] Interval       Transfer     Bandwidth
[ 15]  0.0-10.0 sec  1.25 MBytes  1.05 Mbits/sec
[ 15] Sent 893 datagrams
[ 15] WARNING: did not receive ack of last datagram after 10 tries.
root@mininet-vm:~# 
\end{verbatim}
\end{tiny}

\textbf{Regra de bloqueio de porta: bloquear pacotes com porta de destino 5000}

Devemos mostrar que apenas a porta 5000 está bloqueada:

\begin{verbatim}
iperf do h1 como servidor daemon na porta 5000 e 5001:
\end{verbatim}

\begin{tiny}
\begin{verbatim}
root@mininet-vm:~# iperf -s -p 5000 -D
root@mininet-vm:~# Running Iperf Server as a daemon
The Iperf daemon process ID : 15874

root@mininet-vm:~# iperf -s -p 5001 -D
root@mininet-vm:~# Running Iperf Server as a daemon
The Iperf daemon process ID : 15879

root@mininet-vm:~# 
\end{verbatim}
\end{tiny}

É possível perceber pelas saídas do \texttt{iperf} de \texttt{h2} e \texttt{h3} que ambos os clientes não conseguem conectar-se ao servidor através da porta 5000, apenas pela 5001.

\begin{verbatim}
iperf do h2 como cliente tentando acessar o h1 na porta 5000 e 5001:
\end{verbatim}

\begin{tiny}
\begin{verbatim}
root@mininet-vm:~# iperf -c 10.0.0.1 -p 5000
connect failed: Connection timed out
root@mininet-vm:~# iperf -c 10.0.0.1 -p 5001
------------------------------------------------------------
Client connecting to 10.0.0.1, TCP port 5001
TCP window size: 85.3 KByte (default)
------------------------------------------------------------
[ 15] local 10.0.0.2 port 37974 connected with 10.0.0.1 port 5001
[ ID] Interval       Transfer     Bandwidth
[ 15]  0.0-10.0 sec  3.21 GBytes  2.75 Gbits/sec
root@mininet-vm:~# 
\end{verbatim}
\end{tiny}

\begin{verbatim}
iperf do h3 como cliente tentando acessar o h1 na porta 5000 e 5001:
\end{verbatim}

\begin{tiny}
\begin{verbatim}
root@mininet-vm:~# iperf -c 10.0.0.1 -p 5000
connect failed: Connection timed out
root@mininet-vm:~# iperf -c 10.0.0.1 -p 5001
------------------------------------------------------------
Client connecting to 10.0.0.1, TCP port 5001
TCP window size: 85.3 KByte (default)
------------------------------------------------------------
[ 15] local 10.0.0.3 port 36148 connected with 10.0.0.1 port 5001
[ ID] Interval       Transfer     Bandwidth
[ 15]  0.0-10.0 sec  2.76 GBytes  2.37 Gbits/sec
root@mininet-vm:~# 
\end{verbatim}
\end{tiny}

\section{Configuração dos computadores virtual e real usados nas
medições (se foi usado mais de um, especifique qual passo foi feito
com cada um)}

\begin{itemize}
    \item Computador virtual: Ubuntu (64-bit), 1GB de Memoria RAM e 8GB de armazenamento.
    \item Computador real: Linux Mint (64-bit), Processador Intel Core-i5@2.2GHz, 8GB de Memória RAM e 128GB de armazenamento.
\end{itemize}


\section{Referências}

\begin{itemize}
   \item Kumar, Pradeep. "How to capture and analyze packets with tcpdump command on Linux". \textit{Linux Techi}, 26 Ago. 2018, www.linuxtechi.com/capture-analyze-packets-tcpdump-command-linux/
\end{itemize}

\end{document}
